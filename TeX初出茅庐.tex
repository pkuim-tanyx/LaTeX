\documentclass[UTF8]{ctexart}
\usepackage{amsmath}
\begin{document}
	\title{\textbf{高等数学B(II)期末考试题}}
	\author{子羽翛}
	\date{\today}
	\maketitle
	1.(15分)\\
	求曲线积分\[\oint_{L}(y^2+z^2-2x^2)dx+(z^2+x^2-2y^2)dy+(x^2+y^2-2z^2)dz\]
	其中$L$为球面$z=\sqrt{2Rx-x^2-y^2}$与柱面$x^2+y^2=2Rx$($0<r<R,z\geq0$)的交线,$L^{+}$与球面上侧成右手系.\\
	\\2.(10分)\\
	(1)求二重积分$\iint_{D}\frac{1}{x^2}dxdy$,$D$是由$y=\alpha x,y=\beta x,x^2+y^2=a^2,x^2+y^2=b^2$
	($0<\alpha<\beta<\frac{\pi}{2},0<a<b$)所围的第一象限内的部分.\\
	(2)求锥面$z=\sqrt{x^2+y^2}$被柱面$z^2=2x$所截部分的曲面面积.\\
	\\3.(20分)讨论下列级数的敛散性:\\
	(a)$\sum_{n=1}^{\infty}\frac{\cos{n\phi}}{\sqrt{n}}$($\phi\in(\frac{\pi}{3},\frac{\pi}{2})$)\\
	(b)$\sum_{n=1}^{\infty}\sin{(\pi\sqrt{n^2+1})}$\\
	(c)$\sum_{n=1}^{\infty}\frac{\ln{(1+\alpha^n)}}{n^\beta}(\alpha>0,\beta>0)$\qquad\\
	\\4.(15分)解微分方程:\\
	(a)$3xy'-y-3xy^4$$\ln{x}$=0\qquad
	(b)$y''-9y=e^{3x}\cos{x}$\qquad
	(c)$x^2y''-4xy'+6y=0$\\
	\\5.(20分)\\
	设函数$f(x)$是以$2\pi$为周期的连续函数,$a_{0},a_{n},b_{n}(n=1,2,\cdots)$为其Fourier系数,求函数
	\[F(x)=\frac{1}{\pi}\int_{-\pi}^{\pi}f(t)f(t+x)dt\]的Fourier系数$A_{0},A_{n},B_{n}(n=1,2,\cdots)$.
	\\6.(20分)\\
	(1)证明:如果函数$f(x)$在闭区间$[a,b]$上连续,$g(x)$在闭区间$[a,b]$上不变号且是可积的,则在$[a,b]$上至少存在
	一个点$\epsilon$,使得\[\int_{a}^{b}f(x)g(x)dx=f(\epsilon)\int_{a}^{b}g(x)dx\]\\
	(2)证明:若$f(x)$连续,积分$\int_{A}^{+\infty}\frac{f(x)}{x}dx$对$\forall A>0$都有意义,则有
	\[\int_{0}^{+\infty}\frac{f(ax)-f(bx)}{x}dx=f(0)\ln{\frac{a}{b}}(a>0,b>0)\]\\
	(3)根据(2)的结论,求\[\int_{0}^{+\infty}\frac{\arctan{ax}-\arctan{bx}}{x}dx(a>0,b>0)\].
\end{document}